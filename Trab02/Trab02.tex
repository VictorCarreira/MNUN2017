%%%%%%%%%%%%%%%%%%%%%%%%%~~~RELATÓRIO DE MNUM TRABALHO 01-2017~~~%%%%%%%%%%%%%%%%%%%%%%%
%%%%%%%%%%%%%%%%%%%%%%%%%%%%%%%~~OBSERVATÓRIO NACIONAL~~%%%%%%%%%%%%%%%%%%%%%%%%%%%%%%%%
%%%%%%%%%%%%%%%%%%%%%%%%%%%%%~~ VICTOR RIBEIRO CARREIRA~~%%%%%%%%%%%%%%%%%%%%%%%%%%%%%%%
%------------------------------------------------------------------------------------------------------------------------------------------%
					             %MODELO DE DOCUMENTO%
%-------------------------------------------------------------------------------------------------------------------------------------------------------%
\documentclass[12pt,a4paper,final]{report}%modelo do documento tipo relatório



%-------------------------------------------------------------------------------------------------------------------------------------------------------%
										   %PACOTES UTILADOS%
%-------------------------------------------------------------------------------------------------------------------------------------------------------%
%\documentclass[border = 60pt]{standalone}
%\usepackage[landscape]{geometry}
%\usepackage{tikz}
%\usetikzlibrary{mindmap}
%\usepackage{metalogo}
%\usepackage{dtklogos}

%\documentclass[border=10pt]{standalone}

%%%%%%%%%%%%%%%%

\usepackage{tikz}
\usetikzlibrary{arrows,calc,positioning}

\tikzstyle{intt}=[draw,text centered,minimum size=6em,text width=5.25cm,text height=0.34cm]
\tikzstyle{intl}=[draw,text centered,minimum size=2em,text width=2.75cm,text height=0.34cm]
\tikzstyle{int}=[draw,minimum size=2.5em,text centered,text width=3.5cm]
\tikzstyle{intg}=[draw,minimum size=3em,text centered,text width=6.cm]
\tikzstyle{sum}=[draw,shape=circle,inner sep=2pt,text centered,node distance=3.5cm]
\tikzstyle{summ}=[drawshape=circle,inner sep=4pt,text centered,node distance=3.cm]
%%%%%%%%%%%%%%%%%%%%%%%%%
\usepackage{smartdiagram}
\usepackage[utf8x]{inputenc}
\usepackage{ucs}
\usepackage{multicol}
\usepackage[comma,authoryear]{natbib}%citação com parentesis e autor ano
\usepackage[english, brazil]{babel}
\usepackage{amsmath}
\usepackage{amsfonts}
\usepackage{amssymb}
\usepackage[colorlinks = true, linkcolor = blue, urlcolor  = blue, citecolor = blue, anchorcolor = blue]{hyperref}
\usepackage{indentfirst}
\usepackage{setspace}
\usepackage{makeidx}
%\usepackage[table]{xcolor}
\usepackage{graphicx}
\usepackage{color}
\usepackage{lipsum} % Required to insert dummy text. To be removed otherwise
\usepackage{epstopdf}%adiciona imagens em formato eps no pdf.
\usepackage{subfigure}%cria ambientes de multifiguras
\usepackage{float}%coloca as figuras exatamente aonde você quer
%\usepackage[monochrome]{xcolor}%imprime o arquivo final em preto e branco
\usepackage[left=2cm,right=2cm,top=2cm,bottom=2cm]{geometry}
\usepackage{lipsum} % Required to insert dummy text. To be removed otherwise
\usepackage{multicol, blindtext, graphicx}%cria figura na página inteira
\usepackage{booktabs} % To thicken table lines
\usepackage{tikz}%pacote para fazer fluxogramas
\usepackage{verbatim}%
\usetikzlibrary{calc,trees,positioning,arrows,chains,shapes.geometric,decorations.pathreplacing,decorations.pathmorphing,shapes,matrix,shapes.symbols}
\author{Victor Ribeiro Carreira}
\title{Projeto de Doutorado 2016}


%--------------------------------------------------------------------------------------------------------------------------------------------------------%
								%INÍCIO DO RELATÓRIO%
%--------------------------------------------------------------------------------------------------------------------------------------------------------%

\begin{document}
\thispagestyle{empty}% retira a numeração da primeira página


\begin{figure}[H]
\centering
\subfigure{\includegraphics[scale=1.3]{/home/carreira/Documentos/MNUN2017/Imagens /logoON.jpg}}
%\subfigure{\includegraphics[scale=1.5]{Imagens/logoMCT.png}}
\end{figure}

\vspace{1cm}

\begin{center}
\textbf{MODELAGEM NUMÉRICA DE ONDAS SÍSMICAS}
\end{center}

\vspace{3cm}

\begin{center}
\textbf{RELATÓRIO $02$:} CÁLCULO DA CONSTANTE DE PROPORCIONALIDADE.
\end{center}

\vspace{2.5cm}

\begin{center}
\textbf{PROFESSOR:} LEANDRO DI BARTOLO
\end{center}

\begin{center}
\textbf{ALUNO:} VICTOR RIBEIRO CARREIRA
\end{center}

\vspace{10cm}

\begin{center}
- 2017 -
\end{center}


%--------------------------------------O relatório---------------------------------------------------

\pagebreak%
\setstretch{1.5}
\section*{Introdução}


As técnicas de processamento sísmico fazem uso da equação de ondas compressionais para meios homogênios. Estas ondas são chamadas de ondas acústicas. 

As versões linearizadas desta equação são obtidas através de duas equações básicas, que são a Lei de \textit{Hooke}, e a equação do movimento.   


\section*{Objetivo}

O objetivo deste relatório é calcular a constante de proporcionalidade entre a deformação volumétrica e a pressão hidroestática uniforme. E, durante o processo, utilizar as leis empíricas discutidas, no slide $14$, aula $3$, e o princípio da superposição aplicado a um paralelepípedo imerso em um flúido.



\section*{A problemática envolvida}

Algumas considerações precisam ser feitas para efeito de cálculo. A primeira delas, é que o flúido é considerado um meio isotrópico\footnote{Meio no qual as forças estáticas cisalhantes são nulas.} com viscosidade zero. E a segunda é que a equação da onda acústica é a sua versão linearizada das duas equações básicas citadas no item \textbf{Introdução}.

A onda acústica é definida pelos seguintes termos:

\begin{itemize}
\item[$p$] é a variação de pressão ($N/m^{2}$=$Pa$)
\item[$\vec{v}$] é a velocidade da partícula ($m/s$)
\end{itemize}

A pressão total é indicada pela variável, $p_{t}$, Eq. \ref{ptotal}

\begin{equation}
p_{t}=p_{0}+p
\label{ptotal}
\end{equation}

Onde $p_{0}$ é chamada pressão hidroestática e $p$ representa as mudanças de pressão causada pelo campo de ondas. Desta forma, de maneira similar a desidade total do flúido pode ser definida como na Eq. \ref{denstotal}:

\begin{equation}
\rho_{t}=\rho_{0}+\rho
\label{denstotal}
\end{equation}

A Fig. \ref{fig1} ilustra o problema da superposição.

\begin{figure}[H]
\centering
\includegraphics[scale=0.4]{/home/carreira/Documentos/MNUN2017/Imagens/CPcubo.png}
\caption{Ilustração de um cubo (vermelho) imerso em um flúido.}
\label{fig1}
\end{figure}


\subsection*{O cálculo}



%-----------------------------------Bibliografia-------------------------------------------------%
%\chapter{Referências}
\bibliographystyle{apalike}
\bibliography{referencias.bib}

\end{document}
