\documentclass[16pt,a4paper,final]{report}
\usepackage[latin1]{inputenc}
\usepackage{amsmath}
\usepackage{amsfonts}
\usepackage{amssymb}
\usepackage{makeidx}
\usepackage{indentfirst}
\author{Leandro Di Bartolo}
\title{Prova pr�tica}
\begin{document}
	PROVA PR�TICA DATA DE ENTREGA: 08/09/2017 \\
	 
	Adaptar algoritmo de propaga��o de ondas ac�sticas em Fortran (ou um algoritmo equivalente em Fortran ou qualquer outra linguagem) para a migra��o
	reversa no tempo p�s-empilhamento (zero offset). O programa dever� ser rodado para dado sint�tico simples (de poucas
	camadas e com pequeno contraste de imped�ncia) gerado com o conceito do refletor explosivo, o que � parte integrante do trabalho.
	Assim, as seguintes etapas devem ser realizadas:
	
	 1) Gera��o do dados sint�tico (se��o empilhada) a ser migrado e do modelo
	de velocidades (usado para gerar o dado e na RTM). Sugest�o: adotar um modelo de velocidades de 3 camadas, isto �, com 2 refletores, um plano e paralelo com
	mergulho e outro com curvatura. O programa de modelagem deve ser empregado 	utilizando o modelo de velocidade gerado. A ideia do modelo do refletor explosivo
	deve ser utilizada no programa de modelagem para a gera��o da se��o.
	
	 2) Implementa��o do algoritmo de migra��o RTM zero offset e aplicar o
	algoritmo desenvolvido a se��o gerada no item anterior. Dica: O algoritmo de modelagem deve ser adaptado, sendo o modelo de velocidades gerado e a se��o
	s�smica os dados de entrada.
	
	O trabalho deve ser enviado para leandrodibartolo@gmail.com e deve constar de um: 
	
	(a) relat�rio descrevendo o trabalho, 
	
	(b) do modelo de velocidade gerado e o c�digo fonte utilizado para ger�-lo,
	
	 (c) da se��o 	s�smica sint�tica gerada em conjunto com o c�digo fonte utilizado para 	ger�-la e
	 
	  (d) da se��o migrada com o algoritmo RTM zero Offset, bem como o c�digo fonte utilizado para ger�-la.
\end{document}